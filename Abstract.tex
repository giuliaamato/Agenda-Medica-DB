\documentclass[a4paper,11pt]{report}
\usepackage[T1]{fontenc}
\usepackage[italian,english]{babel}
\usepackage[utf8]{inputenc}
\usepackage[xindy]{imakeidx}
\usepackage{xcolor}
\usepackage{graphicx}
\usepackage{amsmath}
\usepackage{amssymb}
\usepackage{etaremune}
\usepackage{enumitem}
\usepackage[toc,page]{appendix}
\usepackage{verbatimbox}

\usepackage[hidelinks, colorlinks=true]{hyperref}	
\usepackage{bookmark}
\usepackage{caption}
\usepackage{subfig}

\captionsetup{tableposition=top, figureposition=bottom, font=small}
\setcounter{tocdepth}{3}
\setcounter{secnumdepth}{3}


\usepackage{epigraph}


\makeindex


			
\begin{document}

\title{Progetto di Basi di Dati}
\author{Giulia Amato 1075626, Paolo Broglio}
\date{06-2016}

\maketitle


\begin{abstract}
Il progetto consiste nello sviluppare una base di dati che contenga e gestisca le informazioni relative ad un'agenda medica per la gestione delle visite mediche nelle principali ASL del territorio Veneto. Tale agenda potrà essere visualizzata da due tipi di utenti che potranno effettuare delle operazioni tipiche grazie all'utilizzo di un proprio account:
1) Dottori, che hanno la possibilità di creare dei referti e di visualizzarli, visualizzare le visite mediche secondo una tale data e le informazioni dei pazienti
2) Amministratori, che gestiscono la parte amministrativa. Aggiungono nuovi dottori o infermieri, ma possono anche modificarne i dati o eliminarli.

\end{abstract}


\end{document}	
