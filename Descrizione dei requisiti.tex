\begin{document}

\title{Progetto di Basi di Dati}
\author{Giulia Amato, matr. 1075626  Paolo Broglio, matr. XXXXXX}
\date{06/2016}


\begin{Descrizione dei requisiti}
Si vuole realizzare una base di dati che contenga e gestisca le informazioni relative ad un'agenda medica per la gestione delle visite mediche, ed in particolare si vogliono conoscere: dati relativi allo svolgimento della visita medica, dati relativi al paziente e al medico, ed inoltre dati relativi alla ASL di appartenenza. 

Per ordine, una data visita medica è identificata dalle seguenti informazioni:
- Un codice, che la identifica univocamente
- Una data
- Un'ora
- Una priorità con la quale si potranno eventualmente ordinare le visite

Una visita potrà dunque rientrare in due stati (effettuata o prenotata), ed inoltre tale visita potrà essere descritta come visita di controllo o come visita medica.
Al termine di una visita sarà prodotto un referto, dotato di codice che lo identifica univocamente e un testo che descrive l'avvenuta visita. 
Ogni visita verrà svolta in un certo ambulatorio contenuta in una data ASL del territorio del Veneto, tale ASL sarà identificata da un codice univoco, e avrà altri dati significativi quali indirizzo e contatti (email e telefono). Inoltre, la gestione delle prenotazioni di visite mediche per ogni ASL verrà effettuata dal CUP (Centro Unico Prenotazioni), il quale è identificato da un codice univoco e da una password.

Ad ogni ASL afferiscono un certo numero di dottori e infermieri, nonché amministratori di sistema.

L'amministratore e il dottore avranno accesso al database per la gestione delle visite, perciò verranno identificati da:
- Un nome utente, univoco
- Una password
- Una data di scadenza della password, che dovrà essere rinnovata 
- Uno stipendio

Per quanto riguarda il dottore è molto importante essere a conoscenza di dati aggiuntivi quali:
- La specializzazione, affinché le visite di una certa impronta possano essere indirizzate ai dottori competenti su quel settore
- Orario di inizio e fine turni di lavoro, importante per l'inserimento delle visite in un dato range di tempo

Per quanto riguarda l'infermiere, anch'esso avrà come dati significativi lo stipendio, ma anche l'attributo tirocinante.

Tali categorie di utenza avranno tutte una serie di informazioni che caratterizzano un singolo individuo:

- Un codice fiscale, che identifica univocamente un utente
- Un nome ed un cognome
- Una data di nascita
- Il sesso
- Contatti quali email e telefono che identificano univocamente un singolo utente
- Indirizzo, con via e civico
- Città di nascita e città di residenza

Ogni città ha inoltre un nome univoco, una provincia, un CAP e una regione, anch'essa con un nome univoco. 

\end{Descrizione dei requisiti}